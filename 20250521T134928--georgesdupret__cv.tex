%%%%%%%%%%%%%%%%%%%%%%%%%%%%%%%%%%%%%%%%%
% Long Professional Curriculum Vitae
% LaTeX Template
% Version 1.1 (9/12/12)
%
% This template has been downloaded from:
% http://www.latextemplates.com
%
% origin: https://www.latextemplates.com/template/long-professional-cv
%
% Original author:
% Rensselaer Polytechnic Institute (http://www.rpi.edu/dept/arc/training/latex/resumes/)
%
% Important note:
% This template requires the res.cls file to be in the same directory as the
% .tex file. The res.cls file provides the resume style used for structuring the
% document.
%
%%%%%%%%%%%%%%%%%%%%%%%%%%%%%%%%%%%%%%%%%

%----------------------------------------------------------------------------------------
%	PACKAGES AND OTHER DOCUMENT CONFIGURATIONS
%----------------------------------------------------------------------------------------

\documentclass[10pt]{res} % Use the res.cls style, the font size can be changed to 11pt or 12pt here

\usepackage{helvet} % Default font is the helvetica postscript font
%\usepackage{newcent} % To change the default font to the new century schoolbook postscript font uncomment this line and comment the one above
\usepackage{hyperref}

\usepackage{etoolbox} %% remove the asterisk appearing before the bibliography
\patchcmd{\thebibliography}{\section*{\refname}}{}{}{}

\newsectionwidth{0pt} % Stops section indenting
%\renewcommand{\familydefault}{\sfdefault}

\usepackage{lmodern}
\usepackage[T1]{fontenc}
\begin{document}

%----------------------------------------------------------------------------------------
%	YOUR NAME AND ADDRESS(ES) SECTION
%----------------------------------------------------------------------------------------

\name{Georges Dupret\\ \\} % Your name at the top

% If you don't want one of the addresses, simply remove all the text in the first or second \address{} bracket

%\address{{\bf School Address} \\ Department of Study \\ University \\ City, State 12345 \\ (000) 111-1111} % Your address 1

\address{26 Playstead Rd. \#2 \\ Newton, MA-02458 \\ +1 (617) 678-1443 \\ \href{mailto:georges.dupret@gmail.com}{georges.dupret@gmail.com}} 

%----------------------------------------------------------------------------------------

\begin{resume}

%----------------------------------------------------------------------------------------
%	OBJECTIVE SECTION
%----------------------------------------------------------------------------------------

\section{\centerline{EXECUTIVE SUMMARY}}

\vspace{12pt} % Gap between title and text


Accomplished research and data scientist with a strong focus on
organizational and managerial capabilities. Skilled in leading and
coordinating complex projects, with extensive experience in machine
learning (statistical modeling, decision trees, LLM, DNN, etc.) and
statistics (AB experiments, causality, testing, etc.). Demonstrated
proficiency in managing project lifecycles, fostering collaboration,
and driving strategic initiatives, supported by robust analytical
skills in mathematics and programming proficiency.

\vspace{8pt}

%----------------------------------------------------------------------------------------
%	PROFESSIONAL EXPERIENCE SECTION
%----------------------------------------------------------------------------------------

\section{\centerline{PROFESSIONAL EXPERIENCE}} 

\vspace{8pt} % Gap between title and text

% -------------------------------------------------- SPOTIFY

{\sl \textbf{Spotify, Boston, Massachusetts, USA}} \hfill November 2021 -- Present \\
Principal Scientist 

\vspace{8pt}

\begin{itemize}
\item I spearheaded a ground-up initiative to develop innovative
  model-free methods for strategic decision-making using historical
  data. My efforts involved evangelizing the concept and actively
  seeking out collaborators to drive adoption and implementation,
  resulting in its increasing use at the highest levels of the
  company~\cite{dupret2024fortune}.
\item Promote and coordinate efforts to implement advanced Large
  Language Model methods to assess the quality of music playlists (LLM
  as a Judge) before their release, ensuring high production
  standards. Develop a Minimum Viable Product to showcase the
  technology, and coordinate various production teams to scale the
  application for use in Tableau reporting and as a support tool for
  editors and annotators.
\item Established a research team to investigate the application of
  Heterogeneous Treatment Effect (HTE) studies in A/B experiments. The
  analyses generated by HTE are typically unstable and challenging to
  interpret. We overcame these obstacles, significantly enhancing the
  tool's usability and accessibility for non-specialists.
\end{itemize}

% -------------------------------------------------- APPLE

{\sl \textbf{Apple, Cupertino, California, USA}} \hfill September 2017 -- November 2021 \\
Principal Scientist in Maps Organization working on Ranking, Machine Learning and Data Analysis.

I was part of the team responsible for the search engine powering
address and business searches in Apple Maps. While collaborating on
ongoing maintenance and enhancements with the team, I also
lead several individual initiatives:

\vspace{8pt}

\begin{itemize}
\item Debugging issues within a large and complex system posed
  significant challenges. A notable difficulty with the learning
  algorithm was determining the contribution of different features in
  ranking items. We leveraged Shapley values as a principled solution,
  which we extended to effectively compare result pairs.
\item With the rise of deep learning at that time, we investigated its
  application to improve search rankings. We developed the initial MVP
  for a search system that efficiently utilized embeddings, making it
  suitable for production.
\item Considering the critical role of physical distance in map search,
  along with user implicit feedback, I developed innovative features and
  data structures tailored to address map topology and the high
  dimensionality inherent in user feedback.
\end{itemize}

% -------------------------------------------------- FULLPOWER

{\sl \textbf{Fullpower, Santa Cruz, California, USA}} \hfill August 2016 -- September 2017 \\
Data Science and Machine Learning Director

\vspace{8pt}

Fullpower is the technology leader for IoT Cloud-Connected Digital
Sports, Sleep Monitoring, Smart Home and Connected Objects, powered by
AI, Machine Learning and Data Science.

\begin{itemize}
\item Led transition from rule-based systems to machine learning in
  sleep monitoring solutions.
  
\item Organized data collection and label generation in collaboration
  with sleep specialists, while overseeing the machine learning
  pipeline development.
  
\end{itemize}


{\sl \textbf{AliveCor, San Francisco, California, USA}} \hfill December 2014 -- May 2016 \\
Principal Scientist

\vspace{8pt}

\begin{itemize}
\item Developed advanced algorithms for automated analysis and
  diagnosis of heart conditions using AliveCor's FDA-approved
  electrocardiogram device.

\item Reorganized the entire codebase in C\texttt{++} to prioritize testing and
  reproducibility, ensuring compliance with FDA standards.

\item Created tools using R Shiny for efficient triaging of
  misdiagnosed ECG cases.

\item Engineered specialized algorithms for signal filtering, beat
  classification, and pathology detection.

\item Defined priorities and set the agenda for the Science Team in
  collaboration with Project Managers to align with organizational
  goals.
\end{itemize}

{\sl \textbf{Yahoo! Labs, Sunnyvale, California, USA}} \hfill October 2008 -- December 2014 \\
Senior Scientist, Machine Learned Ranking Group

\vspace{8pt} 

\begin{itemize}
\item Developed and optimized Web and Vertical Search ranking
  algorithms utilizing click-through data analysis, leading to new
  interpretations that enhanced search engine results globally. This
  work earned a Best Paper Award at WSDM2010.
  
\item Innovated user engagement metrics across Yahoo properties,
  balancing revenue generation with improved user experience through
  localized and global metrics.
  
\item Designed computationally efficient methods for query
  recommendations, improving the relevance and efficiency of search
  results.
  
\item Advanced the ranking algorithm's sensitivity to document
  recency, boosting the freshness and relevance of search outputs.
\end{itemize}

{\sl \textbf{Yahoo! Research Latin America, Santiago, Chile}} \hfill March 2006 -- October 2008 \\
Researcher

\vspace{8pt}

\begin{itemize}
\item Developed novel evaluation metrics for structured documents
  using principled user model assumptions.
  
\item Managed and supervised multiple dissertations and master's
  theses, contributing to academic and professional advancements in
  the field of web research.
\end{itemize}

{\sl \textbf{Center for Web Research, Universidad de Chile, Santiago, Chile}} \hfill March 2004 -- March 2006 \\
Research in Information Retrieval and Text Mining:

\vspace{8pt}

\begin{itemize}
\item Pioneered statistical models for the explicit modeling of user
  click behavior within search engine log data, laying foundational
  work in understanding search engine interactions.
  
\item Conducted advanced Principal Component Analysis to optimize
  document representation dimensionality within concept space,
  enhancing text mining capabilities.
  
\item Taught graduate courses including Experimental Design and
  Analysis, Statistics for Engineers, Data Mining of web click-through
  data, and Information Retrieval, contributing to the academic growth
  of engineering students.
\end{itemize}

{\sl \textbf{IBM, Z\"urich Research Laboratory, Switzerland}} \hfill January 2001 -- March 2003 \\
Research Staff Member.\\
Research and application in Information Retrieval and Text Mining:

\vspace{8pt}

\begin{itemize}
\item Applied machine learning techniques to high-dimensional spaces,
  focusing on the extraction and automatic clustering of issues within
  Quality Assurance reports for various IBM external projects.
 
\item Developed an innovative method to determine the optimal number
  of singular values in Latent Semantic Analysis, applied to corpus
  visualization and query formulation.
\end{itemize}

{\sl \textbf{IBM, Tokyo Research Institute, Tokyo, Japan}} \hfill 1998 -- 2000 \\
Internship.\\ 

\vspace{8pt}

\begin{itemize}
\item Identified optimal data storage methods and implemented them
  using C++, enhancing data retrieval efficiency for large databases.
  
\item Automated thesaurus construction and keyword clustering
  processes, and devised novel Singular Value Decomposition
  approximations using Artificial Neural Networks.
\end{itemize}

% ----------------------------------------------------------------------------------------
 
\vspace{0.2in} % Some whitespace between sections


%----------------------------------------------------------------------------------------
%	EDUCATION SECTION
%----------------------------------------------------------------------------------------

\section{\centerline{EDUCATION}} 

\vspace{8pt} % Gap between title and text

{\sl Ph.D.}, 
Policy and Planning Sciences, Tsukuba University \\ 
Thesis - `Constrained Architecture Neural Network and its Application to Data Analysis.'
 
{\sl Master Degree}, 
Policy and Planning Sciences, Tsukuba University \\ 
Dissertation - `Density of Population Analysis Using Artificial Neural Networks.'

{\sl Engineer in Applied Mathematics, - Economics Oriented}, 
Catholic University of Louvain, Belgium
Dissertation - `Central Place Theory and Multipurpose Trips.' at Technical University of Lisbon.

%----------------------------------------------------------------------------------------

\vspace{0.2in} % Some whitespace between sections

%----------------------------------------------------------------------------------------
%	LANGUAGES SECTION
%----------------------------------------------------------------------------------------

\section{\centerline{LANGUAGES}}

\vspace{8pt} % Gap between title and text

Fluent in French, English, Spanish, and Portuguese; proficient in
daily German and Japanese conversation.

%----------------------------------------------------------------------------------------

\vspace{0.2in} % Some whitespace between sections

%----------------------------------------------------------------------------------------
%	PUBLICATIONS
%----------------------------------------------------------------------------------------

\section{\centerline{SELECTED PUBLICATIONS}}

\vspace{8pt} % Gap between title and text

For a comprehensive list, visit my
\href{https://scholar.google.com}{Google Scholar} page (more than two
thousand citations).

Publication~\cite{dupret2010user} is a formal statistical model of
user behavior on a SERP applied to
metrics. Publication~\cite{dupret2010model} won best paper
award\footnote{\url{http://www.wsdm-conference.org/2010/}}
and~\cite{dupret2008user} an Honorable Mention at the SIGIR Test of
Time
Awards\footnote{\url{https://sigir.org/awards/test-of-time-awards/}}. Absence
Time~\cite{dupret2013absence} shows how survival analysis can solve
complex problem in estimating user satisfaction. Finally,
Publication~\cite{dupret2024fortune} is a novel data driven method to
take decision applicable when AB testing is not feasible.
\vspace{0.1in}

% \renewcommand{\refname}{\makebox[\textwidth]{\bf{Selected Papers}}}
\renewcommand{\refname}{\bf{Selected Publications}}
\bibliographystyle{ieeetr}
\bibliography{cvGeorgesDupret}


% ----------------------------------------------------------------------------------------

\end{resume}

\end{document}
%%% Local Variables:
%%% mode: LaTeX
%%% TeX-master: t
%%% End:
